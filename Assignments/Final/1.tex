	\سؤال{}





	\textbf{چه تفاوتی بین درس مهندسی نرم‌افزار و تحلیل طراحی سیستم‌ها وجود دارد؟ چه مباحثی در زمینه این درس باید در ادامه حیات مهندسی خود بیاموزید؟ آیا شما برای مهندسی نرم‌افزار گزینه‌ی مناسبی هستید؟ چرا؟}
	


\begin{itemize}
	\item 
	الف)‌ درس مهندسی نرم‌افزار و تحلیل و طراحی سیستم‌ها موضوعاتی هستند که در دو رشته‌ی مختلف درس داده می‌شوند. در مهندسی نرم‌افزار تمرکز بر روی \underline{تولید و توسعه نرم‌افزار }است؛ این در حالی است که تمرکز درس تحلیل و طراحی سیستم‌ها بر روی ساخت یک \underline{سیستم اطلاعاتی} است. یک سیستم اطلاعاتی نمی‌تواند بدون آن‌که نرم‌افزار آن را بسازیم، ایجاد شود. علت این‌که در زمینه مهندسی نرم‌افزار در مقایسه با تحلیل و طراحی سیستم‌ها درس‌های بیش‌تری ارائه می‌شوند، «کیفیت» در این حوزه است. در واقع در درس تحلیل و طراحی سیستم‌ها بیش‌تر با روش‌گان‌ها\footnote{methodology} سروکار دارد. به زبان دیگر اگر به منابع\footnote{resources} و کتاب‌های این دو نگاهی اجمالی بیندازیم، متوجه می‌شویم که در کتاب‌های تحلیل و طراحی سیستم‌ها به دنبالی روشی برای طراحی مراحل \footnote{phases} سیستم‌های اطلاعاتی هستیم اما در کتاب‌های مهندسی نرم‌افزار بیش‌تر مدیریت پروژه نرم‌افزاری و ارزیابی کیفیت خروجی پروژه مد نظر است.
	\item 
ب) 	مهندسی نرم‌افزار شامل یک فرآیند، مجموعه‌ای از روش‌ها، تمرین‌ها و مجموعه‌ای از ابزارها است که به متخصصان اجازه می‌دهد، نرم‌افزارهای کامپیوتری با کیفیت بالا بسازند.

به همین دلیل، مهندس نرم‌افزار یه مهارت‌های زیادی در حوزه‌های مختلف نیاز دارد که مهم‌ترین ‌آن‌ها عبارتند از:
	\begin{itemize}
		\item
چند کاری\footnote{Multitasking}: به عنوان یک مهندسی نرم‌افزار نیاز است تا توانایی مدیریت کردن چندین پروژه را در محیط ددلاین محور را داشته باشیم. این مهارت‌ را اگر به‌طور جزئی‌تر بخواهیم تحت نظر قرار دهیم شامل 
		\begin{itemize}
			\item سازمان‌دهی
			\item اولویت‌بندی
			\item ددلاین‌ها
			\item مدیریت توقعات
		\end{itemize}
	می‌شود. 
	\item کار تیمی:
	ساخت سیستم‌های نرم‌افزاری عموما یک تلاش انفرادی است اما یک مهندس نرم‌افزار هنوز باید توانایی برقراری ارتباط با افراد و تیم‌های دیگر را داشته باشد و این ارتباط را به‌طور مرتب برقرار کند. به عنوان یک مهندس نرم‌افزار باید قادر باشیم تا نیازهای پروژه را بیان کنیم و اگر با چالش یا مشکلی مواجه شدیم، بحث کنیم. این بخش شامل 
		\begin{itemize}
			\item همکاری
			\item سازش\footnote{compromising}
			\item گوش کردن به‌طور فعال
			\item حل اختلاف
			\item ارتباط شفاهی
			\item ارتباط نوشتاری
		\end{itemize}
			می‌باشد.
	\end{itemize}
	\begin{itemize}
		\item 		توجه به جزییات:
		در صنعت، روش‌ها و استانداردهای خاصی وجود دارد که یک مهندس نرم‌افزار باید به آن‌ها تسلط کامل بیابد. این موارد شامل اصول اولیه، مانند استفاده از سیستم‌های کنترل نسخه (مانند گیت‌هاب) است تا کار قدیمی خود را از دست ندهیم. البته این موارد شامل چیزهای پیچیده‌تری مانند تحلیل و نگهداری کد قدیمی‌تر یا تولید روش‌های استراتژیک برای ایجاد یک چارچوب کدزنی می‌شود. بنابراین مهارت‌های لازم در این قسمت شامل
		\begin{itemize}
			\item مهارت‌های تحلیل
			\item اشکال‌زدایی
			\item مستندسازی فنی
			\item فرمول‌ها
			\item خلاقیت
			\item تفکر بحرانی
		\end{itemize}
	می‌باشد.
	\item 
	زبان‌های برنامه‌نویسی: اکثر مهندسان نرم‌افزار به یک یا دو زبان برنامه‌نویسی تسلط و اشراف کامل دارند و این یک حوزه اصلی تخصص آن‌ها می‌شود. 
	\end{itemize}
\item 
پ) با توجه موارد گفته شده در قسمت «ب»، بنده خود را یک گزینه‌ی مناسب برای مهندسی نرم‌افزار می‌دانم؛ زیرا 
	\begin{enumerate}
		\item تجربه‌ی کار در صنعت به عنوان توسعه‌دهنده و مدیر پروژه را دارم.
		\item با متدولوژی‌های گوناگونی مانند \lr{UP}، \lr{Agile} و... آشنا هستم.
		\item درس‌های زیادی را در دانشکده‌ی مهندسی کامپیوتر با موضوع مهندسی نرم‌افزار و نمره‌ی عالی در این زمینه گذرانده‌ام.
		\item تعداد زیادی از مهارت‌های عمومی لازم و نه فنی (نظیر کار تیمی، آشنایی با ددلاین‌ها و...) را دارا هستم.
	\end{enumerate}
\end{itemize}