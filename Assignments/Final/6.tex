\سؤال{}

\textbf{در فصل هفتم کتاب اصولی کاربردی معرفی شده است. ۵ اصل از این اصول که برایتان جالب است و فکر می‌کنید در آینده مورد استفاده قرار خواهید داد را نام برده و کاربرد آن را برای خودتان توضیح دهید.}

از آن‌جایی که این اصول در ۵ دسته آورده شده‌اند، ترجیح می‌دهم برای پاسخ به این سوال و مطالعه‌ی همه‌ی اصول، از هر دسته، یک مورد را انتخاب کرده و توضیح دهم.
\begin{enumerate}
	\item 
	\textbf{دسته ارتباط - اصل شنیدار }
	
	باید سعی کنیم که تمرکز خود را به کلمات  کسی که صحبت می‌کند، به جای فکر کردن به نحوه‌ی بیان جوابمان به آن صحبت‌ها، قرار دهیم.
	\item 
	\textbf{دسته برنامه‌ریزی - اصل واقع‌گرایی}
	
	 باید این واقعیت را بپذیریم	 که آدم‌ها همه‌ی طول روز را کار نمی‌کنند و بازده صددرصد معنایی ندارد. همیشه اختلال‌هایی وارد روابط هر انسانی می‌شود. غفلت و ابهام از واقعیت‌های زندگی هستند و اتفاق نیز رقم می‌خورد. حتی بهترین مهندسین نرم‌افزار هم اشتباه می‌کنند. این واقعیات باید هنگام برنامه‌ریزی برای پروژه لحاظ شوند.
	\item 
	\textbf{دسته مدل‌سازی - اصل هدف قرار دادن تولید نرم‌افزار و نه مدل}
	
	 چابکی\footnote{agility} به معنای رساندن نرم‌افزار به مشتری در سریع‌ترین زمان ممکن است. مدل‌ها ارزش ساخته شدن دارند اما اگر باعث کند شدن فرآیند نرم‌افزار بشوند باید از وسواس به خرج دادن در مورد آن‌ها پرهیز کرد.
	 \item
	 \textbf{دسته تولید و ساخت - اصل اعتبارسنجی}
	  	\begin{itemize}
	  		\item در صورت لزوم باید کد را بررسی کرده و مورد کنکاش قرار دهیم.
	  		\item \lr{unit test} انجام دهیم و خطاهایی که بعد از این آزمون نمایان می‌شوند را برطرف کنیم.
	  		\item 
	  		کد را بازآرایی\footnote{refactor} کنیم.
	  	\end{itemize}
	 \item 
	 \textbf{دسته مستقرسازی - نرم‌افزارهای مشکل‌دار\footnote{buggy} اول باید درست  و سپس تحویل داده شوند.}
	 
	 زمانی که از لحاظ زمانی تحت فشار هستیم، بعضی از شرکت‌های نرم‌افزاری نرم‌افزارهای با کیفیت کم و این وعده که مشکل‌ها را در نسخه‌ی بعدی برطرف می‌کنند، تحویل می‌دهند. این یک اشتباه بسیار بزرگی است. یک جمله‌ی معروف در نرم‌افزار  هست که می‌گوید: «مشتری چند روزی را که به خاطر تحویل دادن نرم‌افزار با کیفیت بالا تاخیر می‌کنید، فراموش می‌کند اما هرگز مشکلات نرم‌افزار کم کیفیت را فراموش نمی‌کند و نرم‌افزار این موضوع را هر روز به آن‌ها یادآوری می‌کند.»
\end{enumerate}