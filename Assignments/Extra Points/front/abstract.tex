
% -------------------------------------------------------
%  Abstract
% -------------------------------------------------------


\pagestyle{empty}

\شروع{وسط‌چین}
\مهم{چکیده}
\پایان{وسط‌چین}
\بدون‌تورفتگی
همان‌طور که بیش‌تر آدم‌های داخل صنعت نرم‌افزار می‌دانند، تفاوت‌های مشخص و واضحی بین تست دستی و تست اتوماتیک نرم‌افزار وجود دارد. تست دستی نیاز به زمان و تلاش‌هایی دارد تا مطمئن شویم که منبع کد نرم‌افزار همان کارهایی را که موظف است، به‌درستی انجام می‌دهد. علاوه‌براین، تست‌کننده‌های دستی همه‌ی چیزهایی را که پیدا می‌کنند ضبط و ذخیره می‌کنند. این کار شامل چک کردن فایل \lr{log}ها، سرویس‌های خارجی و پایگاه‌داده‌ای از خطاها می‌باشد. 
تست اتوماتیک همان‌طور که از نامش پیداست، بدون دخالت انسانی صورت می‌گیرد. در مقایسه با تست دستی، چون تست اتوماتیک با استفاده از ابزار‌های خودکارسازی انجام می‌گیرد، وقت کم‌تری نیاز است تا تست‌ها انجام بگیرد اما زمان بیش‌تری برای نگهداری افزایش پوشش تست‌ها نیاز است. فرآیند تست دستی می‌تواند بسیار وقت‌گیر و تکراری باشد. تست خودکار برای پروژه‌های بزرگ ،به دلیل نیاز به تست کردن یک ناحیه به‌طور مکرر، بسیار مناسب است.

\پرش‌بلند
\بدون‌تورفتگی \مهم{کلیدواژه‌ها}: 
خودکارسازی، تست، نرم‌افزار
\صفحه‌جدید
