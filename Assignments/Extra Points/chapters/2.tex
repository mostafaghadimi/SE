\فصل{پیاده‌سازی تست خودکار}

در این فصل به نحوه‌ی پیاده‌سازی تست خودکار می‌پردازیم. 

\قسمت{جامعیت تست خودکار}

موقع ارزیابی یک راه‌حل تست، داشتن یک ابزار که مناسب نیاز‌های همه‌ی اعضای تیم فرآیند تست، بسیار ضروری است. این افراد عبارتند از:

\شروع{فقرات}
\فقره \مهم{آزمون‌گرهای دستی}:‌ ضبط و اجرای دوباره یک کار حیاتی برای آزمون‌گرهای دستی است. مخصوصأ آن دسته که در تست خودکار تازه‌کار هستند. توانایی استفاده از اسکریپت‌های ضبط شده یکسان با داده‌های مختلف می‌تواند به صورت دستی حین شناسایی و رفع مشکلات از طریق چندین محیط انجام شود.

\فقره \مهم{مهندسان خودکارسازی}: برای مهندسان خودکارسازی، پشتیبانی قدرتمند از زبان‌های اسکریپت‌نویسی، ادغام با سیستم‌های \lr{CI} و توانایی مقیاس‌پذیر کردن تست‌ها می‌تواند مهم باشد.
\فقره \مهم{توسعه‌دهندگان}: پیاده‌سازی تست‌ها در فرآیند توسعه نیازمند توانایی برای انجام دادن تست‌ها درون \lr{IDE}های مختلف نظیر \lr{Visual Studio Code}
\پایان{فقرات}

\قسمت{تصورهای غلط رایج مربوط به تست خودکار}

\شروع{شمارش}
\فقره \مهم{خودکارسازی وقت آزاد بیش‌تری فراهم می‌کند.}
\\
این تصور که تست خودکار وقت آزاد بیش‌تری به ما می‌دهد هم درست است و هم غلط.
در تست دستی، بیش‌تر زمان برای اکتشاف و تست کردن کارایی اختصاص می‌یابد که باید به طور دستی به دنبال خطاها بگردیم. یک‌بار که این کار انجام می‌شود، آزمون‌گر دستی باید مکررأ این کارها را اجرا کند. اما استفاده از تست خودکار باعث می‌شود که این زمان به شدت کاهش یابد. کار آزمون‌گرهای تست خودکار این است که زمان‌شان را صرف کد زدن تست‌ها و اعمال بهبود روی آن تست‌ها به صورت مکرر است. در اصل، زمان صرف‌شده برای کارهای روزمره و تکراری آزمون‌گر دستی برای آزمون‌گر خودکار صرف تمرکز روی موضوعات بزرگ‌تر و مهم‌تری که شامل نرم‌افزار در حال توسعه می‌شود، می‌باشد.

\فقره \مهم{هزینه‌ی تست خودکار بسیار بالا است.}
\\
در ابتدا، سرمایه‌گذاری روی تست اتوماتیک ممکن است هزینه‌بر به نظر برسد، به خصوص اگر صاحب یک شرکت کوچک باشیم. اما تحلیل‌ها نشان داده است که با گذشت زمان، تست خودکار هزینه‌ی خودش را می‌پردازد (هزینه‌ی اضافی ندارد). همان‌طور که قبلا اشاره شد، تست اتوماتیک شما را آزادتر می‌کند تا بتوانید روی مسائل مهم‌تر مانند نیاز‌های مشتری، کارایی و بهبودها تمرکزکنید. هم‌چنین تست خودکار هزینه‌ها و نیاز به بازبینی برنامه را کاهش می‌دهد. علاوه‌براین هر بار که منبع کد تغییر پیدا می‌کند، تست‌های نرم‌افزار می‌تواند تکرار شود. تکرار این تست‌ها به صورت دستی هزینه‌بر است و زمان زیادی را مصرف می‌کند اما تست خودکار می‌تواند بدون هزینه‌ی اضافی تکرار شود.

\فقره \مهم{تست اتوماتیک بهتر از تست دستی است.}
\\
واقعیت این است که بهتر و بدتری وجود ندارد. هر کدام از راه‌کارها خوبی‌ها و بدی‌های خودشان را دارند. تست دستی توسط یک نفر که جلوی کامپیوتر نشسته است به کمک بررسی دقیق \lr{log}هاُ، تلاش برای دادن ورودی‌های مختلف، مقایسه‌ی نتیجه با رفتار مورد انتظار و ضبط نتایج انجام می‌شود.  این در حالی است که تست خودکار معمولأ وقتی نسخه‌ی اولیه‌ی نرم‌افزار توسعه داده شده است، انجام می‌گیرد. 

\فقره  \مهم{تست خودکار از تعامل انسانی جلوگیری می‌کند.}
\\
یکیدیگر از تصورهای غلط درباره تست خودکار این است که تعامل انسانی را تضعیف می‌کند. به طور دقیق‌تر، تست‌های اتوماتیک سریع‌تر از توانایی انسان می‌تواند آزمایش‌ها را انجام دهد با این مزیت که بدون خطای انسانی است.  . بنابراین این تصور قابل درک است. 
\\
البته این موضوع جایگزین روابط چهره به چهره و ملاقاتی که برای قسمت‌های توسعه نرم‌افزار ضروری است، نمی‌باشد. در عوض، با ارائه‌ی یک کانال دیگر ارتباطی، این جنبه را بهبود می‌بخشد. برای مثال ایمیل جای‌گزین تلفن نشد و صرفا یک ابزار اضافی برای برقراری ارتباط بود.

\پایان{شمارش}

\قسمت{تشریح چارچوب‌های تست خودکار}

\شروع{شمارش}

	\فقره \مهم{چارچوب خودکارسازی داده‌محور}
	\شروع{فقرات}
	\فقره 
	\textbf{تعریف} 
	با استفاده از چارچوب داده محور، داده‌های تست را از منطق اسکریپت جدا می‌شوند، به این معنی که مهندسان تست می‌توانند داده‌ها را در یک منبع جدا ذخیره کنند. خیلی اوقات، مهندسان تست در شرایطی قرار می‌گیرند که باید چندین بار ویژگی‌ها یا عمل‌کردهای یک برنامه را با مجموعه‌های مختلف داده تست کنند. در این موارد، بسیار مهم است که داده‌های آزمون در خود اسکریپت \lr{hard-code} نباشند، این اتفاقی است که با یک چارچوب تست خطی یا مبتنی بر ماژولار رخ می‌دهد.
	
	تنظیم چارچوب تست داده‌محور به مهندسان تست این امکان را می‌دهد تا پارامترهای ورودی/خروجی را آزمایش کند و اسکریپت‌ها را از یک منبع داده خارجی مانند \lr{Excel Spreadsheets}، \lr{Text Files}، \lr{CSV}، جداول  \lr{SQL} یا مخازن \lr{ODBC} ذخیره کند.
	
	اسکریپت‌های تست به منبع داده خارجی متصل شده و در صورت لزوم داده‌های لازم را می‌خوانند.
	
	
	\فقره 
	\textbf{مزایا} 
	تست‌ها را می‌توان با مجموعه داده‌های متعدد اجرا کرد.
	
	با تغییر داده‌ها می‌توان به سرعت چندین سناریو را آزمایش کرد و از این طریق تعداد اسکریپت‌های مورد نیاز را کاهش داد.
	
	از \lr{hard-code} کردن داده‌ها می‌توان جلوگیری کرد بنابراین هرگونه تغییر در اسکریپت‌های آزمایشی روی داده‌های مورد استفاده تاثیر نمی‌گذارد و برعکس.
	
	با اجرای سریع‌تر تست‌های بیش‌تر، در وقت صرفه‌جویی می‌شود.
	
	
	\فقره
	\textbf{معایب}
	
	برای استفاده صحیح از این چارچوب، به یک مهندس تست باتجربه و دارای مهارت در زبان‌های مختلف برنامه‌نویسی نیاز دارید. آن‌ها باید منابع داده خارجی را شناسایی و قالب‌بندی کنند و کدی را بنویسند (ایجاد توابع) که آزمایش‌ها را به صورت یک‌پارچه به منابع داده خارجی متصل می‌کند.
	
	تنظیم یک چارچوب داده‌محور زمان زیادی را می‌طلبد.

	
	\پایان{فقرات}
	
	\فقره \مهم{چارچوب خودکارسازی کلیدواژه‌محور}
	\شروع{فقرات}
	\فقره 
	\textbf{تعریف}
	در یک چارچوب کلید‌واژه‌محور‌، هر عمل‌کرد برنامه مورد تست در یک جدول با یک سری دستورالعمل به ترتیب متوالی برای هر تست که باید اجرا شود، قرار داده شده است. با روشی مشابه چارچوب داده محور، داده‌های تست و منطق اسکریپت در یک چارچوب کلیدواژه‌محور از هم جدا می‌شوند، اما این رویکرد آن را یک قدم فراتر می‌برد.
	
	با این رویکرد، کلمات کلیدی نیز در یک جدول داده خارجی ذخیره می‌شوند و باعث می‌شوند که آن‌ها از ابزار تست خودکار که برای اجرای تست‌ها استفاده می‌شود، مستقل شوند. کلمات کلیدی بخشی از یک اسکریپت است که نمایان‌گر اقدامات مختلفی است که برای تست \lr{GUI} یک برنامه انجام می‌شود. این موارد می‌توانند به سادگی با \lr{"click"}، یا \lr{"login"} یا برچسب‌های پیچیده مانند \lr{"clicklink"}، یا \lr{"verifylink"} برچسب‌گذاری شوند.
	
	در جدول، کلمات کلیدی به صورت مرحله به مرحله با یک آبجکت مرتبط یا بخشی از رابط کاربری که عمل در آن انجام می‌شود، ذخیره می‌شوند. برای این‌که این روی‌کرد به درستی کار کند، یک مخزن آبجکت مشترک برای یافتن اقدامات مرتبط با هر آبجکت لازم است.
	
	جدول
	
	
	
	\فقره
	\textbf{مزایا}
	دانش برنامه‌نویسی حداقل مورد نیاز است.
	
	یک کلمه کلیدی واحد را می‌توان در چندین اسکریپت تست استفاده کرد‌، بنابراین کد قابل استفاده مجدد است.
	
	اسکریپت تست را می‌توان مستقل از نرم‌افزار تحت آزمون ساخته شده است.
	
	\فقره
	\textbf{معایب}
		هزینه اولیه تنظیم چارچوب زیاد است. وقتٰگیر و پیچیده است. کلمات کلیدی نیاز به تعریف دارند و باید مخازن/کتاب‌خانه‌های آبجکت تنظیم شود.
	
	شما به یک مهندس تست با مهارت‌های اتوماسیون بالا نیاز دارید.
	
	هنگام مقیاس دادن به عملیات تست، کلمات کلیدی می‌توانند به دردسر تبدیل شوند. شما نیاز به ادامه ساخت مخازن و جداول کلمات کلیدی دارید.
	
	\پایان{فقرات}
	
	\فقره \مهم{چارجوب خودکارسازی ماژولار}
	
	\شروع{فقرات}
	\فقره 
	\textbf{تعریف} 
	اجرای یک چارچوب ماژولار به مهندسین تست نیاز دارد تا برنامه مورد آزمایش را به واحدها ، کارکردها یا بخش‌های جداگانه تقسیم کنند که هر یک از آن‌ها به صورت جداگانه تست می‌شوند. پس از شکستن برنامه به ماژول‌های جداگانه، یک اسکریپت تست برای هر قسمت ایجاد می‌شود و سپس برای ساخت تست‌های بزرگ‌تر به صورت سلسله مراتبی ترکیب می‌شود. این مجموعه‌های تست بزرگ‌تر شروع به نمایش تست کیس‌ها می‌کنند.
	
	یک استراتژی اساسی در استفاده از چارچوب ماژولار ایجاد یک لایه انتزاع است، به طوری که هرگونه تغییر در بخش‌های خاص بر ماژول اصلی تاثیر نمی‌گذارد.
	
	
	\فقره 
	\textbf{مزایا} 
	در صورت ایجاد هرگونه تغییر در برنامه، فقط باید ماژول و آن اسکریپت تست مرتبط با آن تغییر یابد، به این معنی که لازم نیست بقیه تست‌ها را تغییر دهید و می‌توانید آن‌ها را دست نخورده بگذارید.
	
	ایجاد موارد آزمایشی تلاش کم‌تری می‌کند زیرا می‌توان از اسکریپت‌های آزمون برای ماژول‌های مختلف استفاده مجدد کرد.
	\فقره
	\textbf{معایب}
	داده‌ها از آن‌جا که تست‌ها به طور جداگانه انجام می‌شوند  به صورت \lr{hard-code} در تست‌ها قرار دارند، بنابراین نمی‌توانید از چندین مجموعه داده استفاده کنید.
	
	دانش برنامه نویسی برای تنظیم این چارچوب نیاز است.
	
	
	\پایان{فقرات}

	
	\فقره \مهم{چارچوب خودکارسازی ترکیبی}

	\شروع{فقرات}
\فقره 
\textbf{تعریف}
مانند اکثر فرایندهای تست امروز، چارچوب‌های تست خودکار شروع به یک‌پارچه‌سازی و هم‌پوشانی با یک‌دیگر می‌کنند. یک چارچوب ترکیبی، ترکیبی از هر یک از چارچوب‌های قبلی است که برای استفاده از مزایای برخی و کاهش نقاط ضعف برخی دیگر تنظیم شده است.

هر برنامه متفاوت است، بنابراین باید فرآیندهای تست آن‌ها نیز متفاوت باشد. با حرکت تیم‌های بیش‌تر به یک مدل چابک ، تنظیم یک چارچوب انعطاف‌پذیر برای تست خودکار بسیار مهم است. یک چارچوب ترکیبی می‌تواند به آسانی با نیاز ما سازگار شود تا بهترین نتایج تست را به دست آوریم.

\پایان{فقرات}


\پایان{شمارش}