
\فصل{مفهوم تست خودکار}

در این فصل نکات کلی و برخی مفاهیم در خودکارسازی فرآیند تست نرم‌افزار به اختصار توضیح داده می‌شود.

\قسمت{‌خودکارسازی تست چیست؟}

مفهوم فرآیند خودکارسازی تست‌ها به استفاده از ابزارهای خاصی برای کنترل اجرای تست‌ها و مقایسه‌ی نتایج آن‌ها با نتایج مورد انتظار می‌گویند. ابزارهای تست نه‌تنها به ما کمک می‌کنند تا تست‌های رگرسیون را انجام دهیم، بلکه کمک می‌کند تا تولید داده‌ها، نصب محصول، تعامل با رابط گرافیکی و... خودکار شود.

\قسمت{ملاک‌های انتخاب ابزار}
برای خودکارسازی هر برنامه‌ای، باید به موارد زیر توجه کنیم:



\شروع{فقرات}

\فقره قابلیت داده‌محور بودن
\فقره عدم وابستگی به پلتفرم
\فقره قابلیت اجراپذیری و شخصی‌سازی
\فقره اعلان ایمیل
\فقره قابلیت استفاده از \lr{ٰVersion Control}
\فقره قابلیت عیب‌یابی و \lr{logging}
\فقره و...
\پایان{فقرات}

\قسمت{انواع چارچوب‌ها}

به طور معمول، چهار چارچوب خودکارسازی تست وجود دارد که در حین خودکارسازی برنامه‌ها انتخاب می‌شوند:

\شروع{فقرات}

\فقره چارچوب خودکارسازی داده‌محور
\فقره چارچوب خودکارسازی کلیدواژه‌محور
\فقره چارجوب خودکارسازی ماژولار
\فقره چارچوب خودکارسازی ترکیبی

\پایان{فقرات}


\قسمت{تست کیس}

تست کیس یک سندی است که دارای مجموعه‌ای از داده‌های تست، پیش‌شرط‌ها، نتایج مورد انتظار و پس‌شرط‌ها است که برای یک سناریو تست مشخص طراحی شده تا یک نیاز مشخص را تایید و اعتبارسنجی کند.

\قسمت{طراحی تست‌ کیس}

در قسمت زیر، تکنیک‌های متداول طراحی تست کیس در مهندسی نرم‌افزار آورده شده است.

\شروع{شمارش}
\فقره استخراج تست کیس‌ها به‌طور مستقیم از نیازمندی‌های تعریف شده یا تکنیک طراحی جعبه سیاه. این تکنیک شامل:
	\شروع{فقرات}
	\فقره تجزیه و تجلیل مرز مقادیر
	\فقره افراز به قسمت‌های برابر
	\فقره جدول تصمیم‌گیری تست
	\فقره نمودار انتقال حالت
	\فقره تست موارد استفاده
	\پایان{فقرات}
است.



\فقره استخراج تست کیس‌ها به طور مستقیم از ساختار یک جزء یا سیستم.
\شروع{فقرات}
\فقره پوشش \lr{Statement}
\فقره پوشش \lr{Branch}
\فقره پوشش \lr{Path}
\فقره تست \lr{LCSAJ}
\پایان{فقرات}

 \فقره استخراج تست کیس‌ها براساس تجربه‌ی آزمون‌گر روی سیستم‌های مشابه یا شهود.

\شروع{فقرات}
\فقره خطا در حدس‌زدن
\فقره تست اکتشافی
\پایان{فقرات}
\پایان{شمارش}
